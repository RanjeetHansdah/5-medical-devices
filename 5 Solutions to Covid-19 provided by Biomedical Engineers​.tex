\documentclass{article}[A4,12pt]
\usepackage{graphicx}
\usepackage[english]{babel}
\usepackage[utf8]{inputenc}

\begin{document}



\begin{centering}
\huge
\title\bf{ NIT RAIPUR}\\~\\


\includegraphics[scale=0.5]{National_Institute_of_Technology,_Raipur_Logo} \\~\\


\begin{LARGE}
\bf 5 Solutions to Covid19 provided by Biomedical Engineers\\~\\
\end{LARGE}

\end{centering}

\large By: Ranjeet Hansdah\\

\large Roll.no: 21111045\\

\large Branch: Bio-Medical Engineering\\~\\

\newpage

\section*{Introduction}
Biomedical engineering is the term used for the combination of biology and engineering or applying engineering materials to medicine and healthcare.Biomedical engineers are very important for the healthcare industry – from advancing medical treatments to monitoring a condition without them the healthcare industry would be very unreliable and unstable.Most biomedical engineers works in scientific research, pharmaceutical companies, and manufacturing firms.COVID-19 is caused by the SARS-CoV-2 virus. COVID-19 can cause mild to severe respiratory illness, including death. Now we gonna see how biomedical engineers tried to tackle the Covid-19 pandemic.

\section{Ventilators}
COVID-19 can causes respiratory symptoms like coughing, trouble breathing, and shortness of breath.A ventilator has the lifesaving task of supporting the lungs. These machines can provide air with an elevated oxygen content and create pressure in the lungs to assist with breathing. They also help clear away carbon dioxide and rebalance the blood’s pH levels.The ventilator can either partially or fully take over the breathing process for you.A ventilator can help save the lives of some people with COVID-19 by supporting their lungs until their bodies can fight off the virus.This technology has proven to be very efficient to restore the oxygen saturation levels in critically ill patients with COVID-19.So biomedical engineers had to make and provide this ventilators to varies hospitals as there was becoming shortage of ventilators.

\section{Oxygen}
The most common symptom is dyspnea, which is often accompanied by hypoxemia. Patients with severe disease typically require supplemental oxygen and should be monitored closely for worsening respiratory status, because some patients may progress to acute respiratory distress syndrome (ARDS).Infected and damaged lungs are less effective at allowing oxygen to pass from the environment to the bloodstream. The main reason for being admitted to hospital with COVID-19 is to receive supplemental oxygen, to increase the amount of oxygen in the lungs and blood, which will be sufficient treatment before recovery in most cases. This can be administered in a number of ways, including into the nose using plastic tubing, or via a loose-fitting face mask.the oxygen comes in the form of cylinders, either small for portability or large for stationary patients and longer-term supply.Oxygen concentrators extract oxygen from the air on demand and supply it directly to the patient.Engineers had to make this oxygen cylinders due to its high demand in different treatment processes.

\section{Continuous Positive Airway Pressure (CPAP)}
If breathing extra oxygen isn’t enough to improve the oxygen level in the blood, oxygen under pressure can be used to help the movement of gases in and out of the lungs. This is given via a tightly-fitting mask connected to a machine via plastic tubing. The patient remains awake, and doctors can control the pressure and amount of oxygen delivered by the machine. However, this treatment requires large quantities of oxygen, which may be limited in hospitals that are treating large numbers of COVID-19 patients. Again this treatment required the oxygen containers and engineers had to manufacture them.

\section{Telemedicine}
The development of software and technological applications, such as telemedicine to track the evolution of the virus in the population, has gained attention due to the risk of infection and rapid spread.Locating infected people being a key aspect to stop the progress of the virus—authorities have taken preventive measures, isolating the relatives of a person diagnosed positive with COVID-19, in addition to contacting people who were close to the affected individual in the days prior to diagnosis.For this, digital tools have been developed facilitating the work of governments. Digital monitoring or contact tracking applications have different approaches; sometimes, they use the integrated Global Positioning System (GPS) of the phones to geographically locate people infected with the virus and alert the residents of the sector; other developers use the phone’s Bluetooth to connect it with the closest devices, to alert the presence of someone with COVID-19 symptoms or at risk of infection .

\section{Protective suits and equipments}
When COVID-19 pandemic started it demanded the medical supplies and emergency care equipment, with special emphasis on personal protective equipment.There was production of large-scale personal protective equipment using 3D printers from face shields to masks and biosafety suits.This protective equipment has different characteristics including the potential to remove viruses from the surface through antiviral coatings. This technology has been also used to fabricate hand-free door openers, disinfection equipment, emergency fans, air filters, robots that disinfect high-risk areas, wireless sensor to detect COVID-19 early symptoms.Since many healthworkers were in the risk of getting covid they need medical eqiupments to protect themselves better from this virus while also tending to patients.With proper suits and facemaskes they could not have given treatment to the patients.So its demand also rised and engineers had to make innovation in it to improve it.

\section*{Conclusion}
In this way Biomedical engineers were working continously behind the scene for the healthcare system.It is thanks to biomedical engineering that people are living longer – as a result of the technology and modifications it brings about.Biomedical engineers are in high demand in order to keep on top of the safety trends which this virus is causing globally. As a result, there is a very strong sense that this industry will continue to go from strength to strength.

\end{document}