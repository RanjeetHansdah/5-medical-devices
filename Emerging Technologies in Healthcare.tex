\documentclass{article}[A4,12pt]
\usepackage{graphicx}
\usepackage[english]{babel}
\usepackage[utf8]{inputenc}

\begin{document}



\begin{centering}
\huge
\title\bf{ NIT RAIPUR}\\~\\


\includegraphics[scale=0.5]{National_Institute_of_Technology,_Raipur_Logo} \\~\\


\begin{LARGE}
\bf Emerging Technologies in Healthcare \\~\\
\end{LARGE}

\end{centering}

\section*{}

\large By: Ranjeet Hansdah\\

\large Roll.no: 21111045\\

\large Branch: Bio-Medical Engineering\\~\\
\newpage


\section*{Introduction}
Emerging technologies in the healthcare industry are being introduced at a rapid pace, bringing with them the promise of improved treatment options and more efficient care. This is especially important for healthcare facilities that are seeking solutions to deal with staffing shortages or other limitations.Health equity is all about making healthcare more accessible and affordable for everyone — and digital health trends are the driving forces behind health equity.Here are some new emerging technologies in healthcare. 

\section{3d Bioprinting}
The invention of 3D printing is another new technology in the healthcare industry that is proving to be transformative.Recently, a research team developed a method to 3D-print living skin and blood vessels. This is a great breakthrough for skin grafts for burn victims.With the help of DNA analysis, bio-printing can regenerate and replace several body parts, bones, and tissue.Medical devices can now be perfectly matched to the exact specifications of a patient, and be compatible with their natural anatomy. A patient’s body is more likely to accept implants, prosthetics, and devices when they’re perfectly aligned and customized, and the patient often expresses greater comfort and improved performance outcomes as a result.This new field of 3D Bioprinting enables physicians to print artificial limbs, organs, joint replacement parts, and bio tissues. In addition, in the field of pharmacology, there are ongoing experiments for printing pills and other medications. Lastly, 3D printers can also help create medical devices and surgical tools.


\section{Blockchain}
This technology is expected to completely transform the collection and storage of medical history. Not only it would be easier to store information and access it through blockchain, but security threats would also be minimized. It would allow doctors to access the entire medical history of a patient, including any genetic illnesses and allergies, allowing them to customize treatment to provide the best possible care. Compounding the problem is the idea that sharing medical data between facilities, and between scientists, could vastly quicken the development of effective treatments, but the propensity to share such data is dampened by the fear of a security breach. The magic bullet for all these concerns, many experts think, is blockchain technology.Built around a system of currently unhackable cryptography, blockchain tech keeps a distributed ledger of vast amounts of information. This not only securely stores data, but it cuts out the middleman and saves costs for providers and patients alike. Blockchain can also reduce the possibility of fraud.The concept of blockchain for healthcare is still under development.

\section{5G Enabled Devices}
If the biggest drivers of cutting-edge technology—AI, IoT, and Big Data—are to reach their full potential in healthcare, they need a reliable and lightning-fast internet connection. Enter 5G. With a reliable real-time connection, the most immediate benefits will be seen in telemedicine, expanding access to care for millions. But that’s only the beginning. More connected devices, with more authentic data streams, open up the possibility of a revolutionized healthcare system.With next-to-zero latency, 5G-connected sensors and medical devices can capture and transmit data nearly instantaneously. That will improve patient monitoring, which will in turn improve patient outcomes. Futurists are already considering the benefits between 5G, healthcare, and robotics.But patients won’t have to wait long to see a change: experts say 5G-enabled devices will rapidly bring on a new healthcare system.5G in medical device technology will provide more medical gadgets to patients who can reliably measure and monitor their health from home.With 5G-enabled wearable devices, healthcare providers can monitor patients remotely and gather real-time data for preventative care and other individually-tailored healthcare provisions.


\section{Augmented Reality and Virtual Reality}
Augmented and virtual reality (AR and VR) have several applications in the medical world. Simulated and hybrid environments have found a natural fit in medical education, providing simulation training that enhances and works alongside traditional school. Immersive learning with AR and VR headsets can cater to several different learning styles at once by engaging the full range of senses: audio, visual, and kinesthetic.VR can be used in physical therapy to help patients recover from complex limb injuries and even has applications in areas like mental trauma where it can alleviate phobias and PTSD through customized exposure and treatment.In healthcare, VR helps with surgical training and planning, enabling both surgeons and patients to get more comfortable with procedures. There are also many reports about the efficacy of VR for helping with chronic pain management and mental health.These technologies transcend the short shelf life of many digital health trends, as they present tangible long-term benefits for medical professionals and patients, which can save time and money, and improve overall patient care.VR is still a developing technology. As it continues to advance, its capabilities herald exciting developments for areas like preventive healthcare, rehabilitation, and cancer therapy.


\section{Robotic Surgery}
Robots were used medically for the first time in 1985. Since then, their role has expanded in this sector. Robots are mostly used in surgeries and to some extent in procedures such as laparoscopy, neurosurgery, orthopedic surgery, emergency response, and minimally invasive operations. The technology also has wearables such as robotic prostheses or exoskeletons that help with mobility issues in those suffering from missing or paralyzed limbs. Additionally, robotics are deployed for mundane tasks like restocking, disinfecting, cleaning, etc.Robotic surgery is minimally invasive, more precise, less prone to infection, and quicker to heal.They can shape a bone to precisely fit a prosthetic with the accuracy a human never could. Robots also gaining popularity as companions for the elderly or child patients.Healthcare robots are also poised to take over clerical and routine tasks to free up nursing and other healthcare professionals to focus more on direct patient care.


\end{document}