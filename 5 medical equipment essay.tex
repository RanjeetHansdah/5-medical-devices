\documentclass[12pt,A4paper]{article}
\usepackage{graphicx}
\begin{document}
\begin{center}
\huge\textbf{5 Medical Devices}\\[10pt]
\large{By Ranjeet Hansdah}\\
\large{Bio-medical Engineering}\\
\large{Roll.no-21111045}\\
\large{NIT Raipur}\\
\end{center}
\newpage
\section{Dermatoscope}

Dermatoscope also known as dermatoscopy,epiluminescence microscopy, or skin surface microscopy is a non-invasive,in-vivo technique and a hand held device that has been useful for inspection of skin lesions unobstructed by skin surfaces.It helps dermatologists in identifying lesions and differentiate melanocytic lesions from dysplastic lesions,melanomas or non-melanoma skin cancers such as basal cell carcinoma or squamous cell carcinoma.It is also being utilized for diagnosis of skin disorders like inflammatory skins,pigmentary skins,infectious skin and also in examination of hair,skin and nails

Skin surface microscopy started in 1663 by Kolhaus and it was improved by Ernst Abbe in 1878 with the addtion of immersion oil.The German dermatologist, Johann Saphier,who added a built-in light source to the instrument. Goldman was the first dermatologist to coin the term "dermascopy" and to use the dermatoscope to evaluate pigmented cutaneous lesions.In 1989, dermatologists from the Ludwigs-Maximilian-University of Munich developed a new device for dermoscopy. A team of physicians led by Professor Otto Braun-Falco in collaboration with the medical device manufacturer HEINE Optotechnik developed a new dermatoscope.

A dermatoscope consists of a magnifier(10x to 200x),a light sorce(polarized or non polarised),a transparent plate and sometimes a liquid between the instrument with power supply like batteries or with rechargeablility.The basic principle of dermoscopy is transillumination of a lesion in order to study it with high magnification to visualize subtle features.A hand-lens with in-built illumination,cannot allow visualization beyond the surface of the skin because skin may reflect,refract,diffract and/or absorb the incident light.Aplication of a linkage or immersion fluid is used over the skin which enhances translucency and improves the skin visibility.Polarized light allows for visualization of deeper skin structures,while non-polarized light provides information about the superficial skin.Most modern dermatoscopes allow the user to toggle between the two mides,which provides complementary information.The newest generation of dermatoscopes include inbuilt crossed-polarizers,which filter out scattered light from the periphery,reduce glare and permit visualization of skin surface without the need of a linkage fluid.Some dermatoscopes have an inbuilt photography system with supporting software for the capture and storage of images.

Dermoscopy is performed by either the non-contact or the contact technique. In the contact technique, the glass plate of the instrument touches the lesion through the linkage fluid. In the non-contact technique, the cross-polarized lens absorbs all the scattered light and allows only light in a single plane to pass through it without contact of the lens with the skin. The contact technique gives better illumination and resolution. The advantage of the non-contact technique is the prevention of inter-patient infections.Avoidance of cross-infection in the case of contact dermoscopy is by using a barrier like a cling film or adhesive tape over the lesional skin.As a non-invasive technique, dermoscopy is essentially free of complications.Dermoscopy may result in confirmation of clinical diagnosis, often avoiding the need for a skin biopsy. 

Although dermoscopy is an excellent tool for triage, it needs to be combined with the macro clinical picture and histopathology to be conclusive. The role of the dermatopathologist is vital in this regard.The conclusion is that when specialists use dermoscopy, it is a better tool for the diagnosis of melanoma as compared to simple visual examination. Also, dermoscopy is more effective when interpreted with the actual patient, rather than with a dermoscopy image.

Dermoscopy is not only for dermatologists, rather the skill should be acquired and customized by other specialists too, especially general practitioners/family physicians, pediatricians, and dermatosurgeons.
The use of dermoscopy by general physicians is very low.With respect to general/family physicians, regrettably, many barriers have resulted in extremely low usage of dermoscopy. Some of these barriers include - costs of the dermoscopy—both the equipment cost and the relatively inadequate reimbursement for its use in practice, the need for dermoscopy training, lack of information about learning resources and the unwillingness to invest time, both for training and to use dermoscopy in practice.
\newpage 
\section{Electroconvulsive Therapy Unit}
Electroconvulsive therapy (ECT) is a form of psychiatric treatment that involves inducing seizures with the use of electrical stimulation while an individual is under general anesthesia.ECT seems to cause changes in brain chemistry that can quickly reverse symptoms of certain mental health conditions.ECT is most often used for cases of treatment-resistant depression and some other psychiatric conditions including bipolar disorder,mania,catatonia and psychosis.An estimated one million people worldwide have ECT each year.About 70 percent of ECT patients are women.This may be due to the fact that women are more likely to be diagnosed with depression.Older and more affluent patients are also more likely to receive ECT.

Convulsive therapy was introduced in 1934 by Hungarian neuropsychiatrist Ladislas J. Meduna who, believing mistakenly that schizophrenia and epilepsy were antagonistic disorders.The ECT procedure was first conducted in 1938 by Italian psychiatrist Ugo Cerletti.In the US, ECT devices are manufactured by two companies, Somatics, which is owned by psychiatrists Richard Abrams and Conrad Swartz, and Mecta.In the UK, the market for ECT devices was long monopolized by Ectron Ltd, which was set up by psychiatrist Robert Russell.

Usually a course of ECT involves multiple administrations, typically given two or three times per week until the patient is no longer suffering symptoms. ECT is administered under anesthesia with a muscle relaxant.ECT can differ in its application in three ways: electrode placement, treatment frequency, and the electrical waveform of the stimulus. These treatment parameters can pose significant differences in both adverse side effects and symptom remission in the treated patient.Typically, 70 to 120 volts are applied externally to the patient's head, resulting in approximately 800 milliamperes of direct current passing through the brain, for a duration of 100 milliseconds to 6 seconds, either from temple to temple or from front to back of one side of the head.Placement can be bilateral, where the electric current is passed from one side of the brain to the other, or unilateral, in which the current is solely passed across one hemisphere of the brain. 

In unilateral ECT, both electrodes are placed on the same side of the patient's head. Unilateral ECT may be used first to minimize side effects such as memory loss.In bilateral ECT, the two electrodes are placed on opposite sides of the head. Usually bitemporal placement is used, whereby the electrodes are placed on the temples. Uncommonly bifrontal placement is used; this involves positioning the electrodes on the patient's forehead, roughly above each eye.Unilateral ECT is thought to cause fewer cognitive effects than bilateral treatment, but is less effective unless administered at higher doses.High-dose unilateral ECT has some cognitive advantages compared to moderate-dose bilateral ECT while showing no difference in antidepressant efficacy.The electrodes deliver an electrical stimulus. The stimulus levels recommended for ECT are in excess of an individual's seizure threshold.

Aside from effects on the brain, the general physical risks of ECT are similar to those of brief general anesthesia.Immediately following treatment, the most common adverse effects are confusion and transient memory loss.ECT can cause a lack of blood flow and oxygen to the heart, heart arrhythmia and other heart problems.Some patients experience muscle soreness after ECT.Among treatments for severely depressed pregnant women, ECT is one of the least harmful to the fetus.In adolescents, ECT is highly efficient for several psychiatric disorders, with few and relatively benign adverse effects.

No one knows for certain how ECT helps treat severe depression and other mental illnesses. What is known, though, is that many chemical aspects of brain function are changed during and after seizure activity. These chemical changes may build upon one another, somehow reducing symptoms of severe depression or other mental illnesses.The exact mechanism of ECT still remains unknown.

ECT often works when other treatments are unsuccessful and when the full course of treatment is completed, but it may not work for everyone.Much of the stigma attached to ECT is based on early treatments in which high doses of electricity were administered without anesthesia, leading to memory loss, fractured bones and other serious side effects.ECT is much safer today. Although ECT may still cause some side effects, it now uses electric currents given in a controlled setting to achieve the most benefit with the fewest possible risks.

\newpage
\section{Nephroscope}[\textwidth14pt]
Nephroscopy is a procedure to examine the inside of your kidney and to treat certain conditions in your upper urinary tract.Nephroscopy is performed using a small instrument called a nephroscope. Nephroscopy is a non-invasive procedure that reduces the need for traditional open surgery.The thin, tube part of the nephroscope is inserted into your skin through a very small cut.

The nephroscope was introduced by the Storz company in 1965 (Pearson, 1975) was the first instrument used for viewing the pyelocaliceal system, and the nephroscope with optical fibers and round lenses was for a long time the standard instrument for renal percutaneous surgery.

 The nephroscope has channels within it that provide a source of light, a telescope and an irrigation system.It has an external sheath and the working element. This contains the optical system and a working channel located in the central axis of the nephroscope, which allows insertion of the lithothripter or different working elements (stone forceps, extraction probe, etc.). The same axial channel also ensures the irrigation fluid’s flow. The fluid returns through the space between the nephroscope and its sheath.The nephroscope uses ultrasound or a laser probe to break apart the target (for example, kidney stones). Once broken apart, the pieces are suctioned out through one of the channels of the scope or pulled out through the scope with graspers.

Nephroscopy is also used to- remove kidney stone fragments,remove small tumors,remove foreign bodies, such as a stent that was previously placed,remove kidney cysts (a fluid-filled pouches on or in your kidney),treat ureteropelvic junction (UPJ) obstruction (a blockage in an area of your kidney called a the renal pelvis) and to check to see if a previous PCNL(percutaneous nephrolithotomy or PCNL)procedure was successful and no kidney stones or fragments remain.

It is performed by a small catheter which is placed through urethra into the kidney. Dye is then injected into the catheter and X-rays are taken to show the inside details of the kidney. Then a needle is inserted through the skin on the back into the kidney at the location determined in the pre-surgery planning. Then the area is dilated and a sheath (tube) is inserted through a dime-sized incision. The sheath allows the nephroscope – and other surgical instruments that are inserted through the nephroscope – direct access to the inside of the kidney. The target of the procedure (for example, kidney stone or blockage) is broken up and/or removed through the nephroscope.And also an urethral stent is also placed in the kidney. A stent is a 10- to 12-inch soft, hollow plastic tube that is positioned the full length of the ureter.The stent holds the ureter open, which helps the body drain urine and encourages the kidney to heal. This stent is usually removed within week after the procedure completed.

Risks include serious bleeding, fluid buildup in the lungs, sepsis, urinary tract infection and injury to the ureter or kidney.

Nephroscopy is a very safe procedure that reduces the need for traditional surgery, which involves a longer recovery time and greater risk of infection.

Technological progress from recent years has materialized in the development of digital nephroscopes which increase the ergonomics of the instruments used.Globally, the new generation of digital nephroscopes offers a more ergonomic design and a higher image resolution compared with those that are based on optical fibers.

\newpage
\section{Revascularization Laser}
Like every other organ or tissue in your body, the heart muscle needs oxygen-rich blood to survive. The heart gets this blood from the coronary arteries. But in patients with coronary artery disease (CAD), the coronary arteries are clogged and diseased and can no longer deliver enough blood to the heart. The heart’s lack of oxygen-rich blood is called ischemia.Not getting enough oxygen to the heart muscle increases the risk of heart attack and may cause a painful condition called angina.Most of the time, the best treatment for angina is coronary artery bypass surgery. But for some patients with very serious heart disease or other health problems, bypass surgery may be too dangerous. Also, some patients may have had many coronary artery bypass operations and be unable to have more bypass operations.For patients who cannot have bypass surgery, there is a procedure called transmyocardial laser revascularization, also called TMLR or TMR. TMLR cannot cure CAD, but it may reduce the pain of angina.TMR, or transmyocardial laser revascularization, is a new treatment aimed at improving blood flow to areas of the heart that were not treated by angioplasty or surgery. A special carbon dioxide (CO2) laser is used to create small channels in the heart muscle, improving blood flow to the heart muscle.

TMR is a surgical procedure,but it can be done while the heart is still beating and full of blood. That means that a heart-lung machine is not needed. Also, surgeons do not cut open the chambers of the heart, so TMLR is not open heart surgery.. It is performed through a small incision in either the left side or the middle of the chest. Frequently, it is performed along with coronary bypass surgery, occasionally alone.Once the incision is made, the surgeon exposes the heart muscle. A laser hand piece is then positioned on the area of the heart to be treated. A special high-energy, computerized CO2 laser is used to create between 20 to 40 one-millimeter-wide channels (about the width of the head of a pin) in the left ventricle (left lower pumping chamber) of the heart. The doctor determines how many channels to create during the procedure. The outer areas of the channels close, but the inside of the channels remain open inside the heart to improve blood flow.The CO2 Heart Laser uses a computer to direct laser beams to the appropriate area of the heart in between heartbeats, when the ventricle is filled with blood and the heart is relatively still. This helps to prevent electrical disturbances (arrhythmias) in the heart.

Doctors aren't sure how TMR improves blood flow to the heart. Clinical evidence suggests blood flow is improved in two ways-1)The channels act as bloodlines. When the ventricle pumps or squeezes oxygen-rich blood out of the heart, it sends blood through the channels, restoring blood flow to the heart muscle.2)The procedure may promote angiogenesis, or the growth of new capillaries (small blood vessels) that help supply blood to the heart muscle.TMR usually takes one to two hours.

TMR is a treatment option for individuals who have severe angina, which limits their daily activities or causes them to wake from pain at night, despite medications or have pre-operative tests that show ischemia (decreased blood supply to the heart muscle) or have a history of previous bypass surgery or angioplasty, and no further intervention is available.

In a study - published by The New England Journal of Medicine (1999), 72 percentage of patients who had TMR experienced an improvement in angina symptoms after 12 months, compared to only 13 percentage of patients who were receiving medications for the treatment of their angina symptoms.
After TMR, some patients feel immediate relief from angina symptoms, while others feel improvement over time. Some patients do not have improved symptoms after TMR, but may have improved activity tolerance.They may also find that they do not need to take as many heart medicines, including nitroglycerin.They also have a lower risk of heart attack.

\newpage
\section{Tonometer}
Tonometry is a diagnostic test that measures the pressure inside your eye, which is called intraocular pressure (IOP). This measurement can help your doctor determine whether or not you may be at risk of glaucoma.Most tonometers are calibrated to measure pressure in millimeters of mercury (mmHg).Glaucoma is a serious eye disease that can eventually lead to vision loss if untreated. In most cases of glaucoma, the fluid that normally bathes and nourishes the eye drains too slowly, causing pressure to build up.Without treatment, the increased pressure can eventually harm your optic nerve and cause vision loss. According to the American Academy of Ophthalmology (AAO), glaucoma is one of the leading causes of blindness in adults over the age of 60.The changes caused by glaucoma are often painless and can progress for years without you noticing. A tonometry test is critical for detecting the changes early.

Types of tonometry-Goldmann Tonometry:
The most common tonometer that eye care practitioners use is the Goldmann applanation tonometer. A Goldmann tonometer is usually attached to a slit lamp microscope. Anesthetic eye drops are instilled into your eyes, followed by a small amount of fluorescein dye. A cobalt blue light then illuminates the flurorescein and the tonometer. A small probe is gently pressed onto your eye, indenting the cornea. The pressure that the cornea pushes back onto the tonometer is measured in millimeters of mercury, giving your eye doctor or healthcare provider a number to record and compare to from year to year.

Non-Contact Tonometry:
Non-contact tonometry (NCT) is commonly referred to as the "air puff" test.Many people prefer this type of tonometry because it does not involve touching the eye. Instead, a gentle puff of air is used to flatten the cornea. Some studies show that NCT tonometry is not as accurate as Goldman tonometry but NCT provides a very useful and speedy way of measuring eye pressure in children or sensitive adults.

Electronic Tonometry:
Electronic tonometry refers to a handheld, mobile device that your eye doctor or practitioner can carry from room to room to check eye pressure. Resembling a writing pen, the mobile tonometer is gently and quickly applied to your cornea. Your healthcare provider will probably obtain about three readings in order to obtain an accurate measurement. Electronic tonometry is not as reliable or as accurate as Goldman tonometry but is extremely handy for a busy practitioner.

Ocular response analyzer:
The ocular response analyser (ORA) is a non-contact (air puff) tonometer that does not require topical anaesthesia and provides additional information on the biomechanical properties of the cornea. It uses an air pulse to deform the cornea into a slight concavity. The difference between the pressures at which the cornea flattens inward and outward is measured by the machine and termed corneal hysteresis (CH). The machine uses this value to correct for the effects of the cornea on measurement.In a population based study in healthy children that compared non-contact IOP measuring tonometer, including ORA and CORVIS with a contact tonometer, GAT, which is a routine instrument for IOP measurement. It was firmly evident that due to significantly low positive or negligible correlation, none of these 2 non-contact tonometers can replace the GAT

According to the AAO, the risk of glaucoma is high if you are over 40 years old ,are Black, Hispanic, or Asian, have a family history of glaucoma, are nearsighted or farsighted, have other chronic eye conditions, have injured your eye in the past, have diabetes, have high blood pressure, have poor blood circulation, have used corticosteroid medications for prolonged periods of time.

The thickness of the cornea affects most non-invasive methods by varying resistance to the tonometer probe. A thick cornea gives rise to a greater probability of an IOP being overestimated (and a thin cornea of an IOP being underestimated), but the extent of measurement error in individual patients cannot be ascertained from the CCT alone.The Ocular Response Analyzer and Pascal DCT Tonometers are less affected by CCT than the Goldmann tonometer. Conversely, non-contact and rebound tonometers are more affected.Corneal thickness varies among individuals as well as with age and race. It is reduced in certain disease and following LASIK surgery.

According to the Glaucoma Research Foundation, the normal eye pressure range is 12 to 22 mm Hg,Tonometry is extremely safe. However, there’s a very small risk that the cornea could be scratche when the tonometer touches your eye. However, even if this happens, your eye will normally heal itself within a few days.


\end{document}